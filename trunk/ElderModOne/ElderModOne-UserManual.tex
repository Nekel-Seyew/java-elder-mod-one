\documentclass[letterpaper,10pt,titlepage]{article}

\usepackage{graphicx}                                        

\usepackage{amssymb}                                         
\usepackage{amsmath}                                         
\usepackage{amsthm}                                          

\usepackage{alltt}                                           
\usepackage{float}
\usepackage{color}

\usepackage{url}

\usepackage{balance}
\usepackage[TABBOTCAP, tight]{subfigure}
\usepackage{enumitem}

\usepackage{pstricks, pst-node}

\usepackage{geometry}
\geometry{textheight=10in, textwidth=7.5in}

%random comment

\newcommand{\cred}[1]{{\color{red}#1}}
\newcommand{\cblue}[1]{{\color{blue}#1}}

\usepackage{hyperref}

\def\name{K.M.D Sweeney}

%pull in the necessary preamble matter for pygments output
\input{pygments.tex}

%% The following metadata will show up in the PDF properties
\hypersetup{
  colorlinks = true,
  urlcolor = black,
  pdfauthor = {\name},
  pdfkeywords = {elder mod one},
  pdftitle = {Elder Mod One User Manual},
  pdfsubject = {Elder Mod One User Manual},
  pdfpagemode = UseNone
}

\parindent = 0.0 in
\parskip = 0.2 in

\begin{document}

\section{Introduction}
\hrule

The Elder Mod One is a retro-style ray-casting engine written in Java and scripted using Python running on the JVM. Consequently, developers can create their games on any platform and then distribute it to any other platform they wish, so long as that platform is powerful enough and has at least JRE-7 installed. Additionally, users can take advantage of the Jython features to add their own content to distributed games.

This document will aid developers in the development of mods for the engine by illuminating all assumptions made by the engine and giving examples of the API.


\newpage
\section{Mod Package Structure}

In the base folder, there should be several items:
\begin{enumerate}
\item ElderModOne.jar
\lib

\section*{Answers}
\begin{enumerate}
\item The main point was to have students experiment and work with processess and process communication.
\item Testing was again through many printf statements and everytime I compiled. If I compiled, I tested it. Where I was getting errors, i would put printfs so I could isolate the exact line that I was failing at. This usually worked, unless I really didn't understand what was going on.
\item I learned that I need even better time management: I need to start on day 1 of obtaining the assignment, not just 5 days before it's due. I need to also finally meet more people in class so I can ask them for help and compare solutions.
\end{enumerate}

\end{document}
